\documentclass[a4paper,12pt]{article}

\usepackage{geometry}
\geometry{a4paper,left=25mm,right=20mm, top=25mm, bottom=25mm}

\usepackage[T1]{fontenc}
\usepackage[utf8]{inputenc} %utf8 latin1
\usepackage{helvet}
\usefont{T1}{phv}{m}{n}
\renewcommand{\encodingdefault}{T1}
\renewcommand{\rmdefault}{phv}
\usepackage{graphicx}
\usepackage{geometry}
\usepackage{fancyhdr}
\usepackage{cite}
\usepackage[nottoc,notbib]{tocbibind}
\usepackage[english,ngerman]{babel}
\usepackage{bibgerm}
\usepackage[numbers,square]{natbib}
\usepackage{url}
\usepackage[intoc]{nomencl}
\usepackage{listings}
\usepackage{picins}
\usepackage{color}
\usepackage{indentfirst}
\usepackage{titlesec}
\usepackage[]{titletoc}
\usepackage{amsmath}
\usepackage{amssymb}
\usepackage{amstext}
\usepackage{amsfonts}

\setlength{\parindent}{0pt}

% Hyperlinks
\usepackage{hyperref} % mu� als letztes Paket aufgef�hrt werden!
\hypersetup{citebordercolor=1 1 1,menubordercolor=1 1 1,linkbordercolor=1 1 1,urlbordercolor=1 1 1}

% Titel der Arbeit
\def\mytitle{Party Licht Steuerung -- Programmentwurf für Lichttechniker mit LabView}

% Art der Arbeit
% [Bachelorarbeit|Studienarbeit|Projektarbeit|Praxisarbeit|...]
\def\myKindOfReport{Studienarbeit}

% Ihr Name
\def\myname{Tim Berger} 

% Abgabedatum
\def\mydate{17. Juni 2011}

% Bearbeitungszeitraum in Wochen
\def\bearbeitungszeitraum{6. Theoriephase}

% Ihre Matrikelnummer
\def\matrikel{115435}

% Ihre Ausbildungsfirma
\def\firma{Kurtz Holding GmbH \& Co. Beteiligungs KG}

% Betreuer der Ausbildungsfirma
\def\betreuer{Dipl.-Ing. (FH) Dirk Schlosser}

% Gutachter des Pr�fungsausschusses
\def\gutachter{Prof. Dr. Wolfgang Funk}

% Seitennumerierung, Kopf- und Fu�zeile
	\fancyhead{} % clear all fields
	\fancyfoot{} % clear all fields
	\fancyhead[R]{\sl\leftmark}
	\cfoot{\thepage}

%\renewcommand{\familydefault}{\sfdefault}

%Wenn die Gliederungsebenen nicht ausreichen =)
\titlecontents{subsubsubsection}[9em]{}{\contentslabel{3.9em}}%
{\hspace*{-1.2em}}{\titlerule*[0.675pc]{.}\contentspage}
 
\makeatletter
\newcounter{subsubsubsection}[subsubsection]
\setcounter{subsubsubsection}{1}
\setcounter{secnumdepth}{4} 
\setcounter{tocdepth}{5} 
\renewcommand{\thesubsubsubsection}{\thesubsubsection.\@arabic\c@subsubsubsection}
 
\titleclass{\subsubsubsection}{straight}[\subsubsection]
\titleformat{\subsubsubsection}{\bf}{\thetitle}{1em}{}[]						
\titlespacing{\subsubsubsection}{0pt}{3.25ex plus 1ex minus 0.2ex}{1.5ex plus 0.2ex}

%-------------------------------------------------------------------------------------------

\begin{document}

\pagenumbering{Alph} %damit die titel seite nicht die seitenzahl 1 hat

%Fussnoten mit Halbklammer: 1)
\makeatletter
\renewcommand*\thefootnote{\@arabic{\c@footnote})}
\makeatother
\sloppy


\begin{titlepage}
\newgeometry{margin=2cm}
\begin{minipage}[t]{\linewidth}  
  \includegraphics[scale=0.6]{Pics/DHBW.png}
  \hspace{10.5cm}
  \includegraphics[scale=0.6]{Pics/Kurtz.png}
\end{minipage}
	    \vspace*{4ex}
  \begin{center}
    {\Large\bf\vspace*{4ex}\mytitle\\}
    {\large\bf\vspace*{4ex}\myKindOfReport\\}
    {\vspace*{4ex}für die Prüfung zum\\Bachelor of Engineering\\}
    {\vspace*{6ex}im Studiengang TIT08I\\an der Dualen Hochschule Baden-Württemberg Mosbach\\}
    {\vspace*{6ex}von\\}
    {\Large\bf\vspace*{2ex}\myname\\}
    {\bf\vspace*{8ex}\mydate\\}
    	\vfill

\begin{tabbing}
	\hspace*{8cm}\=\kill
	Bearbeitungszeitraum: \> \bearbeitungszeitraum \\
	\\
	Matrikelnummer: \> \matrikel \\
	\\
	Ausbildungsfirma: \> \firma \\
	\\
	%Betreuer der Ausbildungsfirma: \> \betreuer\\
	%\\
	Gutachter der DHBW Mosbach:	\> \gutachter \\
\end{tabbing}
    \end{center}\end{titlepage}
\restoregeometry
%Ende Titelblatt
%\addcontentsline{toc}{section}{Kurzzusammenfassung}
\label{chap:Zusammenfassung}
\section*{Zusammenfassung}

In der vorliegenden Arbeit wird der Programmentwurf einer Party-Lichtsteuerung vom Design der Anwendung über die Implementierung bis hin zur Bereitstellung einer lauffähigen Applikation dokumentiert.
Die Entwicklungsumgebung ist LabVIEW. Abschließend findet sich eine Gegenüberstellung der Vor- und Nachteile des verwendeten graphischen Programmiersystems.



%\addcontentsline{toc}{section}{Abstract}
\thispagestyle{empty}
\label{chap:Abstract}
\section*{Abstract}



\newpage

\renewcommand{\baselinestretch}{1.40}\normalsize
\pagenumbering{Roman} % r�mische Numerierung z.B. f�r Inhaltsverzeichnis
%\renewcommand\contentsname{Inhaltsverzeichnis}
\tableofcontents
%\renewcommand{\refname}{Literaturverzeichnis}
\newpage
%\renewcommand{\figurename}{Abbildung}
%\renewcommand{\listfigurename}{Abbildungsverzeichnis}
\listoffigures
\newpage
%\lstlistoflistings
%\newpage
\renewcommand{\nomname}{Abkürzungsverzeichnis}
\makenomenclature
\printnomenclature



%\nomenclature[prefix]{symbol}{description}
\nomenclature{\textbf{NI}}{National Instruments}
\nomenclature{\textbf{LJ}}{Light Jockey (dt. Lichttechniker) }
\nomenclature{\textbf{LabVIEW}}{ Laboratory Virtual Instrumentation Engineering Workbench}
\nomenclature{\textbf{FGV}}{funktionale globale Variable}
\nomenclature{\textbf{VI}}{virtuelles Instrument}
\nomenclature{\textbf{subVI}}{unter Programm eines VIs}
\nomenclature{\textbf{VM}}{virtuelle Maschine}
\nomenclature{\textbf{LLVM}}{Low-Level Virtual Machine}
\nomenclature{\textbf{}}{}
\nomenclature{\textbf{}}{}
\nomenclature{\textbf{}}{}
\nomenclature{\textbf{}}{}
\nomenclature{\textbf{}}{}
\nomenclature{\textbf{}}{}








%\renewcommand{\listtablename}{bla}
%\listoftables
\newpage
\pagestyle{fancy}
\pagenumbering{arabic} % arabische Numerierung z.B. f�r Inhaltsverzeichnis


\section{Einleitung}%2-10

Diese Studienarbeit dokumentiert den Programmentwurf einer Party-Lichtsteuerung, vom Design der Anwendung über die Implementierung bis hin zur Bereitstellung einer lauffähigen Applikation.
Die Programmcode wird mit LabVIEW von National Instruments entwickelt.

\subsection{Aufgabenstellung}
Es ist ein Programm für Lichttechniker zu entwickeln, mit dem eine viel zahl von Scheinwerfer angesteuert werden kann. 
%Dabei soll ein durch den Lichttechniker erstellter Ablauf abgespielt werden.

\subsection{Anforderungen}
Der Light Jockey (LJ) stellt für verschiedene Lichtkanäle Intensität und Farbe ein. Für eine Gruppe von Lichkanälen (Set) kann eine Wartezeit, Überblendungszeit, Nachlaufzeit und Name eingestellt werden. Wählt der LJ die Schaltfläche zum aufnehmen, öffnet sich ein Fenster in dem die gewünschten Parameter übergeben werden. Nach der Bestätigung durch ein Klick auf die OK-Schaltfläche wird das erstellte Set hinten an die Queue angefügt.

Hat der Bediener einige Sets angelegt, wird mit der Abspiel-Schaltfläche das aufgenommene Programm durchlaufen. Ein Set das abgespielt wird wartet die angegebene Zeit, dann wird die Farbe bis zur Intensität über die Überblendungszeit hochgefahren. Jetzt beginnt die Nachlaufzeit, ist diese verstrichen wird mit dem nächsten Set aus der Queue fortgefahren. Der Bediener kann jeder Zeit eine abspielende Queue mit der Stopp-Schaltfläche anhalten.

Über die Menüleiste kann der Bediener mit dem Menüpunkt "`Datei"' die Queue speichern und laden. In beiden Fällen öffnet sich eine Dialogbox in dem nach Speicher- bzw. Ladepfad gefragt wird.

	\begin{figure}%[h!]
	\centering
		\includegraphics[width=\textwidth]{Pics/Oberflaeche001.png}
	\caption{Bedienoberfläche für die Party-Lichtsteuerung}
	\label{fig:ober001}
	\end{figure}



%\subsection{Motivation und Zielsetzung}

%Quelle: \cite{labview-buch01} \\
%Quelle: \cite{internet}
\subsection{Dokumentation}
Der LabVIEW Quellcode  sowie eine Ausführbare Windows-Anwendung mit der notwendigen RunTime als Insatller findet sich auf der beigelegten CD.
Auch enthalten sind die Programmcode ausschnitte als Bilder in hoher Detailansicht.
Und Projekt auf \url{http://code.google.com/p/party-licht-steuerung/}.


\subsection{Aufbau der Arbeit}

Bla Bla Blaaa





\section{LabVIEW als Programmiersprache}
\label{sec:labview}



\section{Implementierung des Projekts}

\subsection{Entwurfsmuster}
Erzeuger / Verbraucher / 

\subsection{Datentypen}

\subsection{Datei Ein- und Ausgabe}

\subsection{---}



\subsection{Grafische Oberfläche}

\subsection{Projekt Design}

\subsection{Test Plan}

\subsection{Dokumentation}

\subsection{Stand-Alone Applikation}



\section{Abschließende Betrachtung}


\pagestyle{plain}

\addcontentsline{toc}{section}{Literaturverzeichnis}
\label{chap:Literaturverzeichnis}

\bibliography{Projektarbeit}{}
\bibliographystyle{geralpha}



\appendix
\renewcommand{\listtablename}{Anhang}
\renewcommand{\lstlistingname}{} 
\renewcommand{\thelstlisting}{Quellcode \arabic{lstlisting}}
\listoftables
\refstepcounter{section}
\definecolor{BackgroundColor}{RGB}{225,225,225,225}	
\lstset{backgroundcolor=\color{BackgroundColor}, numbers=left, frame=single}

%%% Beginn	
	\newpage	
	\subsection{Initialisierungsfunktion}
	\label{a1}
	\addcontentsline{lot}{section}{A.1 Initialisierungsfunktion}	
	\begin{figure}[!h]
	\centering
		\includegraphics[angle=90, height=0.8\textheight ]{Pics/init.png}
	\caption{Initialisierungsfunktion}
	\label{fig:a1}
	\end{figure}
	\newpage
	
	\subsection{Shutdown-Funktion}
	\addcontentsline{lot}{section}{A.2 Shutdown-Funktion}
	\begin{figure}[!h]
	\centering
		\includegraphics[width=\textwidth]{Pics/shutdown.png}
	\caption{Shutdown-Funktion}
	\label{fig:a2}
	\end{figure}
	%\newpage
	


	\subsection{Initialisierung des Front Panels}
	\addcontentsline{lot}{section}{A.3 Initialisierung des Front Panels}	
	\begin{figure}[!h]
	\centering
		\includegraphics[width=\textwidth]{Pics/init-front.png}
	\caption{Initialisierung des Front Panels}
	\label{fig:a3}
	\end{figure}
	%\newpage
	


	\subsection{Auswahl eines Lichtsets aus der Lichterset Queue}
	\addcontentsline{lot}{section}{A.4 Auswahl eines Lichtsets aus der Lichterset Queue}	
	\begin{figure}[!h]
	\centering
		\includegraphics[width=0.35\textwidth]{Pics/front-auswahl.png}
	\caption{Auswahl eines Lichtsets aus der Lichterset Queue}
	\label{fig:a4}
	\end{figure}
	\newpage
	
	\subsection{De-/Aktivieren von Schaltflächen}	
	\addcontentsline{lot}{section}{A.5 De-/Aktivieren von Schaltflächen}	
	\begin{figure}[!h]
	\centering
		\includegraphics[width=0.5\textwidth]{Pics/front-deaktivieren.png}
	\caption{De-/Aktivieren von Schaltflächen}
	\label{fig:a5}
	\end{figure}
	%\newpage

	\subsection{Update der Set-Ablaufliste}
	\addcontentsline{lot}{section}{A.6 Update der Set-Ablaufliste}
	\begin{figure}[!h]
	\centering
		\includegraphics[width=0.5\textwidth]{Pics/front-updateCue.png}
	\caption{Update der Set-Ablaufliste}
	\label{fig:a6}
	\end{figure}
	\newpage	
	
	

	\subsection{Update der Lichtkanäle}
	\addcontentsline{lot}{section}{A.7 Update der Lichtkanäle}	
	\begin{figure}[!h]
	\centering
		\includegraphics[width=\textwidth]{Pics/front-kanale.png}
	\caption{Update der Lichtkanäle}
	\label{fig:a7}
	\end{figure}
	%\newpage	
	
	\subsection{Timing Modul}
	\addcontentsline{lot}{section}{A.8 Timing Modul}	
	\begin{figure}[!h]
	\centering
		\includegraphics[width=\textwidth]{Pics/zeit.png}
	\caption{Timing Modul}
	\label{fig:a8}
	\end{figure}
	\newpage	
	
	\subsection{Abspiel Zustandsautomat - Überblenden}
	\addcontentsline{lot}{section}{A.9 Abspiel Zustandsautomat - Überblenden}	
	\begin{figure}[!h]
	\centering
		\includegraphics[angle=90, height=0.8\textheight ]{Pics/automat-fade.png}
	\caption{Abspiel Zustandsautomat - Überblenden}
	\label{fig:a9}
	\end{figure}
	\newpage	
	
	\subsection{Stopp-Funktion}
	\addcontentsline{lot}{section}{A.10 Stopp-Funktion}	
	\begin{figure}[!h]
	\centering
		\includegraphics[width=\textwidth]{Pics/stop.png}
	\caption{Stopp-Funktion}
	\label{fig:a10}
	\end{figure}
	%\newpage	
	
	\subsection{Speicher-Funktion}
	\addcontentsline{lot}{section}{A.11 Speicher-Funktion}	
	\begin{figure}[!h]
	\centering
		\includegraphics[width=0.6\textwidth]{Pics/speichern.png}
	\caption{Speicher-Funktion}
	\label{fig:a11}
	\end{figure}
	%\newpage		
	
	\subsection{Lade-Funktion}
	\addcontentsline{lot}{section}{A.12 Lade-Funktion}	
	\begin{figure}[!h]
	\centering
		\includegraphics[width=0.6\textwidth]{Pics/laden.png}
	\caption{Lade-Funktion}
	\label{fig:a12}
	\end{figure}
	%\newpage	
	
	\subsection{Fehlerbehandlung}
	\addcontentsline{lot}{section}{A.13 Fehlerbehandlung}	
	\begin{figure}[!h]
	\centering
		\includegraphics[width=\textwidth]{Pics/fehler.png}
	\caption{Fehlerbehandlung}
	\label{fig:a13}
	\end{figure}
	\newpage
	
	
	\subsection{Gesamter Überblick über das Main VI}
	\addcontentsline{lot}{section}{A.14 Gesamter Überblick über das Main VI}	
	\begin{figure}[!ht]
	\centering
		\includegraphics[angle=90, height=0.85\textheight]{Pics/PLS-Main-all.png}
	\caption{Gesamter Überblick über das Main VI}
	\label{fig:a14}
	\end{figure}
	%\newpage	
%	
%	\subsection{Text}
%	\addcontentsline{lot}{section}{A.15 Text}	
%	\begin{figure}[h!]
%	\centering
%		\includegraphics[width=0.1\textwidth]{Pics/front-kanale.png}
%	\caption{Text}
%	\label{fig:a15}
%	\end{figure}
%	%\newpage	
%	
%	\subsection{Text}
%	\addcontentsline{lot}{section}{A.16 Text}	
%	\begin{figure}[h!]
%	\centering
%		\includegraphics[width=0.1\textwidth]{Pics/front-kanale.png}
%	\caption{Text}
%	\label{fig:a16}
%	\end{figure}
%	%\newpage	
%	
%	\subsection{Text}
%	\addcontentsline{lot}{section}{A.17 Text}	
%	\begin{figure}[h!]
%	\centering
%		\includegraphics[width=0.1\textwidth]{Pics/front-kanale.png}
%	\caption{Text}
%	\label{fig:a17}
%	\end{figure}
%	%\newpage	
%	
%	\subsection{Text}
%	\addcontentsline{lot}{section}{A.18 Text}	
%	\begin{figure}[h!]
%	\centering
%		\includegraphics[width=0.1\textwidth]{Pics/front-kanale.png}
%	\caption{Text}
%	\label{fig:a18}
%	\end{figure}
%	%\newpage	
%
%	
%	\subsection{Text}
%	\addcontentsline{lot}{section}{A.19 Text}	
%	\begin{figure}[h!]
%	\centering
%		\includegraphics[width=0.1\textwidth]{Pics/front-kanale.png}
%	\caption{Text}
%	\label{fig:a19}
%	\end{figure}
%	%\newpage	


\pagestyle{empty}

\addcontentsline{toc}{section}{Erklärung}
\label{chap:Erklärung}

%\section{Erklärung}

\vspace*{\fill}

\noindent{\bf\Large Erklärung}

\vspace*{6ex}

\noindent gemäß § 5 (2) der "`Studien- und Prüfungsordnung DHBW Technik"' vom 18. Mai 2009. Ich habe die vorliegende Arbeit mit dem Titel

\vspace*{4ex}

\noindent{\bf\large\mytitle}

\vspace*{4ex}

\noindent selbständig verfasst und keine anderen als die angegebenen Quellen und Hilfsmittel verwendet.

\vspace*{4ex}
\noindent Mosbach, den \mydate\\[3ex]

% eigenhändige Unterschrift
\noindent\myname

\vspace*{\fill} 


gemäß § 5 (2) der „Studien- und Prüfungsordnung DHBW Technik“ vom 18. Mai 2009.
Ich habe die vorliegende Arbeit selbstständig verfasst und keine anderen als die angegebenen
Quellen und Hilfsmittel verwendet.
%%% Local Variables: 
%%% mode: latex
%%% TeX-master: "../myReport"
%%% End: 


\end{document}
